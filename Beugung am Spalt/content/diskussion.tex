\section{Diskussion}

Anhand des Beugungsbildes am Einzelspalt wird deutlich, wo sich das Maximum nullter Ordnung befindet. Bei den Maxima höherer Ordnungen wird das schon schwieriger, da die Höhendifferenzen zwischen Maximum nullter Ordnung und Maximum erster Ordnung, sowie die von den höheren Ordnungen, groß sind. Die Funktion des Beugungsbildes konvergiert für große Winkel rapide gegen null. Zudem erschweren die vorkommenden Unsicherheiten das Erkennen weiterer Maxima, da diese die Differenzen zwischen hohen Intenstäten und niedrigeren Intensitäten durch die Diskrepanzen der Unsicherheiten weiter verringert werden.\\
Das Beugungsbild am Doppelspalt dagegen weist deutliche Maxima auf. Allerdings wird das Maximum nullter Ordnung nicht so deutlich sichtbar, zumindest wird nur mit Sicherheit abgelesen werden können, im welchen Intervall es sich befindet. Daran wird deutlich, dass die Pieks mit den Maxima für höhere Spaltenzahlen diskreter und dünner werden. Der Messbereich für den Einzelspalt hätte größer werden sollen, um die Maxima und Minima noch weiter verdeutlichen zu können, wohingegen es bei höheren Spaltenzahlen nicht nötig ist. Zudem wird die Asymmetrie der Beugungsbilder deutlich, was laut Theorie (siehe \autoref{eq:es}) nicht so ist. Grund dafür könnte zudem sein, dass der Detektor nicht gut genug ist, um dem idealen und symmetrischen Bild nahe zu kommen. Es ist im Experiment nämlich aufgefallen, dass der Detektor den höschsten Maxima, also den nullter Ordnung, an einem Punkt gemessen hat, welcher nicht in der Mitte gewesen ist, sondern etwas weiter weg. Es könnte aber auch an einigen ungerade oder eventuell sogar rissige Stellen am Spalt liegen. Natürlich könnten die Experimentatoren auch etwas fehlerhaft gemessen haben. 

\label{sec:Diskussion}
