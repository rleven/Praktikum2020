\section{Durchführung}
\label{sec:Durchführung}

\subsection{Charakteristik des Zählrohrs}
Für die Aufnahme der Charakteristik wird eine $\beta$-Quelle vor das Geiger-Müller Zählrohr angebracht.
Bei einer Zählrohrspannung von $\SI{320}{\volt}$-$\SI{700}{\volt}$ und einer Integrationszeit von $t=\SI{60}{\second}$ werden in Schritten von $\SI{10}{\volt}$ die Impulse gemessen.
\subsection{Totzeitbestimmung mit Zwei-Quellen-Methode}
Die $^{204}Tl$-Quelle wird nun näher ans GM-Zählrohr gebracht und mit einer Integrationszeit von $t=\SI{120}{\second}$ aufgenommen.
Drauf wird eine zweite Quelle angebracht und erneut gemessen. 
Zuletzt wird nur die zweite Quelle gemessen.
Aus den drei Messwerten lässt sich nach \autoref{eq:tot} die Totzeit bestimmen.
\subsection{Zählrohrstrom} 
Bei einer Integrationszeit von $t=\SI{60}{\second}$ und bei $\SI{350}{\volt}$-$\SI{700}{\volt}$ wird der Zählrohrstrom in Schritten von $\SI{50}{\volt}$ gemessen.
Hieraus lässt sich die Anzahl an freigesetzten Ladungen pro einfallendem Teilchen bestimmen.