\newpage
\section{Auswertung}
\label{sec:Auswertung}

\subsection{Charakteristik des Zählrohrs}

\begin{figure}
  \centering
  \includegraphics{plot.pdf}
  \caption{Darstellung des Geiger-Plateaus.}
  \label{fig:plot}
\end{figure}


In \autoref{fig:plot} sind die Messdaten des Zählrohrs dargestellt.
Die Unsicherheiten der Impulse sind Poisson-verteilt, sodass $\Delta N = \sqrt{N}$.
Die Steigung des Plateaus(blaue Linie) beträgt $1.316\pm0.22\ \text{Imp/V}$.
Das ergibt einer Steigung von ca 131.6 Impulsen pro $\SI{100}{\volt}$.
Die Zählrate im Plateau liegt im Bereich von 10000 und die Integrationszeit bei $t = \SI{60}{\second}$, damit eine Totzeitkorrektur nicht notwendig ist.

\subsection{Bestimmung der Totzeit}
Die gemessenen Werte für $N_1$, $N_2$ und $N_{1+2}$ sind:
\begin{align*}
  N_1 = 96041\ \text{Imp/120s}\\
  N_2 = 76518\ \text{Imp/120s}\\
  N_{1+2} = 158479\ \text{Imp/120s}
\end{align*}

Nach \autoref{eq:tot} liegt die Totzeit bei etwa $957.971\si{\micro\second}$.

\subsection{Bestimmung des Zählrohrstroms}
Die Werte für die Anzahl Z der freigesetzten Ladungen pro eingefallenem Teilchen stehen in \autoref{tab:zzahl}.
Die Unsicherheit der Stromstärke beträgt $\Delta A = \SI{0.05}{\micro\ampere}$.

\begin{table}
  \centering
  \caption{Liste der gegebenen Werte, sowie der freigesetzten Ladungen mit Fehler.}
  \csvreader[tabular=c|c|c|c|c,
  head=false, 
  table head= Spannung $U$ in $\si{\volt}$ & Strom $I$ in $\si{\micro}\text{A}$ & Anzahl Impulse $N$ & Ladungen $Z$ & Fehler von $Z$\\\midrule,
  late after line= \\]
  {tabelle1.csv}{1=\eins, 2=\zwei, 3=\drei, 4=\vier, 5=\funf}{$\num{\eins}$ & $\num{\zwei}$ & $\num{\drei}$ & $\num{\vier}$ & $\num{\funf}$}
  \label{tab:zzahl}
\end{table}

Das ergibt einen nahezu linearen Anstieg der freigesetzten Ladungen, wie in \autoref{fig:elem} zu erkennen ist.

\begin{figure}[htbp]
  \centering
  \includegraphics{element.pdf}
  \caption{Anzahl der freigesetzten Ladungen mit zunehmender Zählrohrspannung.}
  \label{fig:elem}
\end{figure}

Die Unsicherheiten wurden mithilfe des Gauß'schen Fehlerfortpflanzungsgesetzes ermittelt:
\begin{equation}
  y(x\ +\ \Delta x) = y(x)\ +\ \frac{dy}{dx}\cdot \Delta x = y\ +\ \Delta y
  \label{eq:gauss}
\end{equation}