\section{Auswertung}
\label{sec:Auswertung}

Zunächst wird die Wellenlänge des Helium-Neon-Lasers bestimmt. Hierbei beträgt die Hebelübersetzung Ü = 5,046.\\
Mit \autoref{eq:verschiebung} und der Erweiterung um \(\frac{1}{Ü}\) lässt sich die Wellenlänge bestimmen.

\begin{table}
  \centering
  \caption{Tabelle der Messwerte zur Ermittlung der Wellenlänge des Lasers.}
  \label{tab:tab1}
  \begin{tabular}{c | c | c}
    \toprule
    Impulse \(n\) & Abstand \(d\) [\(10^{-2}\) m] & Wellenlänge \(\lambda\) [\(10^{-7}\) m]\\
    \midrule
    3102 & 0.526 & 6.721\\
    3105 & 0.545 & 6.957\\
    3102 & 0.525 & 6.708\\
    3102 & 0.526 & 6.721\\
    3098 & 0.526 & 6.730\\
    \bottomrule
  \end{tabular}
\end{table}

Aus dem arithetischem Mittel ergibt sich für die Wellenlänge 

\begin{equation}
  \lambda = 6.767 \cdot 10^{-7} m.
\end{equation}

Nun wird der Brechungsindex von Luft gemessen. Hierzu sind einige Werte notwendig:\\
Der Normaldruck, welcher \(p_0 = 1.0132\) Bar beträgt, die Normaltemperatur \(T_0 = 273.15\) Kelvin, die Größe der Messzelle mit \(50 \cdot 10^{-3}\) Meter und schließlich die Umgebungstemperatur, also im Grunde genommen die Raumtemperatur (\(293.15\) Kelvin).\\
Mithilfe von \autoref{eq:Druck} wird dann der Brechungsindex berechnet:

\begin{table}
  \centering
  \caption{Tabelle der Messwerte zur Ermittlung des Brechungsindex' von Luft.}
  \label{tab:tab2}
  \begin{tabular}{c | c | c}
    \toprule
    Impulse \(n\) & Druck \(p\) [Bar] & Brechungsindex \(N\)\\
    \midrule
    3102 & 0.1420 & 1.0001880851\\
    3111 & 0.1566 & 1.0001846511\\
    3102 & 0.1587 & 1.0001840045\\
    3103 & 0.1365 & 1.0001889562\\
    3105 & 0.1897 & 1.0001412949\\
    \bottomrule
  \end{tabular}
\end{table}

Der arithetische Mittelwert lautet hier

\begin{equation}
  N_{Luft} = 1.000177398.
\end{equation}

