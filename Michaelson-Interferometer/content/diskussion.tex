\section{Diskussion}
\label{sec:Diskussion}

Besonders auffällig ist die äußerst hohe Präzession bei der Berechnung der Brechungsindizes. Es treten da erst Veränderungen in der vierten Nachkommastelle auf. Der arithemtische Mittelwert der Messungen unterscheidet sich nur um ca. 0,0001 (der Literaturwert beträgt 1,00028). Der Unterschied an sich ist ein Resultat aus systematischen Fehlern, aber auch die Schwierigkeit beim Ablesen seitens der Experimentatoren.\\
Bei der Ermittlung der Wellenlänge des Lasers wird (bis auf den zweiten Wert) ebenfalls eine hohe Präzession deutlich. Der eine herausragende Wert sorgt für eine deutliche Erhöhung des arithemtischen Mittelwerts. Gleichzeitig aber wurde genau wegen solchen Werten sichergestellt, dass mehrere Messungen vorgenommen werden. Auch hier werden systematische Unsicherheiten und Unsicherheiten beim Ablesen eine Rolle spielen.

\section{Literatur}

https://de.wikipedia.org/wiki/Brechungsindex
