\section{Theorie}
\label{sec:Theorie}

In diesem Versuch ist das Licht als eine Welle zu erfassen und zu diskutieren, welche mit der elektrischen Feldstärke 

\begin{equation}
    E(x,t) = E_0 \cos(kx-\omega t - \delta)
\end{equation}

beschrieben werden kann, da diese eine elektromagnetische Welle ist. Dabei ist \(k\) die Wellenzahl, \(x\) der Ort, \(t\) die vergangene Zeit, \(\omega\) die Kreisfrequenz und \(\delta\) ein Phasenwinkel. Hier kann demnach ebenso davon ausgegangen werden, dass zwei miteinander interferierende Lichtwellen durch Superposition beschrieben werden kann. Was da allerdings einfacher gemessen werden kann, ist die Lichtintensität \(I\), welche proportional zum Betragsquadrat der elektrischen Feldstärke ist. Für die Intensität ergibt sich also bei einer Interferenz zweier Wellen mit \(E_0\):

\begin{equation}
    I_{ges} \propto 2(E_0)^2(1+\cos(\delta_2 - \delta_1))
\end{equation}

Der zweite Summand in der Klammer ist der Interferenzterm. Falls dieser Null wird, erreicht die Intensität ihr Maximum und falls dieser Term zu \(\pi\) wird, erreicht die Intensität ihr Minimum. Dies ist genauso bei geraden Vielfachen von \(\pi\) bzw. ungeraden Vielfache. Das Phänomen wird konstruktive- und destruktive Interferenz genannt, bei der sich die Amplituden einer Lichtwelle addieren, oder voneinander subtrahieren, sodass das Licht verschwindet. Dabei ist zu unterscheiden, ob diese Interferenzen örtlich bzw. zeitlich konstant aufrecht gehalten werden. Übliches Licht aus der Taschenlampe ist nicht kohärent, da die Wellen nicht unendlich lang und zudem zeitlich statistisch verteilt sind. Kohärentes Licht, erzeugt mit Lasern beispielsweise, hat die Eigenschaft, dass die Phasendifferenz zweier Lichtwellen überall gleich bleibt. Bei einem Strahlenteiler wird deutlich, dass sich nicht immer Interferenzen beobachten lassen. 

%Zeichnung einfügen

Es können keine Interferenzen mehr auftreten, falls der Wegunterschied der Lichtstrahlen größer als die Länge der Lichtzüge, weil die Lichtstrahlen zu zwei verschiedenen Zeiten den Beobachtungspunkt P erreichen. Von diesem Punkt bis zum nächsten Punkt, an dem das Licht verschwindet, ist die Kohärenzlänge. Sie ist der Abstand zweier Maxima zueinander. Die Kohärenzlänge \(l\) wird definiert als

\begin{equation}
    l = N\lambda,
\end{equation}
wobei \(\lambda\) die Wellenlänge ist und \(N\) der Abstand zweier beobachteten Interferenzmaxima.
\\
Das Teilen des Lichtstrahls erfolgt dann mit dem Michaelson-Interferometer.

%Bild einfügen

Ein Strahl geht dann durch das Material und trifft auf den Spiegel S2. Der andere Strahl wird auf S1 geschossen ohne irgendwelche Materialien auf dem Weg. Beide Strahlen werden reflektiert und treffen in P aufeinander, wobei interferiert wird. Beide kommen hinterher auf den Detektor D zu. Es stellt sich heraus, dass beide Strahlen kohärent sind. Es gilt dabei dieser Zusammenhang zwischen der Verschiebung \(d\), die Anzahl der beobachteten Interferenzmaxima \(z\) und der Wellenlänge \(\lambda\):

\begin{equation}
   2d=z\lambda.
   \label{eq:verschiebung}
\end{equation}

Läuft ein Strahl dann durch eine Schicht Material mit anderem Brechungsindex, so gilt mit der Änderung des Brechungsindex \(\Delta n\):

\begin{equation}
    2b\Delta n = z\lambda,
\end{equation}

wobei \(n\) durch eine Taylorentwicklung nach \(f(\lambda)\) angenähert werden kann:

\begin{equation}
    n = 1+\frac{f}{2} N
    \label{eq:pups}
\end{equation}

Mit 

\begin{equation}
    N(p,T) = \frac{pT_0}{p_0T} N_L
\end{equation}

und 

\begin{equation}
    \Delta n(p,p') = \frac{fT_0}{2p_0T} N_L (p-p')
\end{equation}

ergibt nicht nach Umstellung der \autoref{eq:pups} folgende Formel:

\begin{equation}
    n(p_0,T_0) = 1 + \frac{z\lambda T p_0}{2bT_0 (p-p')}.
    \label{eq:Druck}
\end{equation}