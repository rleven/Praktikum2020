\section{Durchführung}
\label{sec:Durchführung}

\subsection{Aufbau}
Zum Aufbau gehört eine nicht-monochromatische Lampe, dessen Lichtstrahlen durch eine Lochkarte scheinen.
Die Löcher auf der Karte stellen ein \textit{L} dar.\\
Entlang der Strahlungsrichtung sind ein oder mehrere Linsen aufgebaut, sowie am Ende des Aufbaus ein weißer Schirm.

\subsection{Teilversuche}
\subsubsection{Überprüfung des Abbildungsgesetzes}
Um die \autoref{eq:abb} sowie die \autoref{eq:linse} zu überprüfen, wird eine Sammellinse bekannter Brennweite verwendet.
Mittels Variation der Bild- und Gegenstandsweite sollen so neun Messungen gemacht werden, um die Brennweite zu überprüfen.

\subsubsection{Bestimmung einer unbekannten Brennweite}
Die Messungen werden nun wie in 3.2.1 für eine Linse unbekannter Brennweite durchgeführt.

\subsubsection{Bestimmung der Brennweite nach Bessel}
Zur Bestimmung der Brennweite einer Sammellinse nach Bessel, muss der Abstand von der Lochkarte zum Schirm gleich bleiben.
Um ein scharfes Bild auf dem Schirm zu erhalten, können demnach nur zwei Positionen der Linse existieren.
Für diese Positionen gilt: $b_1=g_2$ und $g_1=b_2$
Es sollen je fünf Positionspaare gemessen werden und damit die bekannte Brennweite der Linse überprüft werden.

\subsubsection{Bestimmung der Brennweite eines Linsensystems nach Abbe}
Mittels des Abbildungsgesetzes wird die Brennweite des Linsensystems aus einer Zerstreuungslinse und einer Sammellinse bestimmt.
Dazu werden die Gegenstandsweite $g'$ und die Bildweite $b'$ gemessen, sowie die größe des Bildes auf dem Schirm $B$ und die Größe des \textit{L} auf der Lochkarte.
Es soll über 2 Messreihen die Brennweite des Linsensystems errechnet werden.