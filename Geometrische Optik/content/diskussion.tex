\newpage
\section{Diskussion}
\label{sec:Diskussion}

\subsection{Überprüfung der Linsengleichung}
Der berechnete Wert für die Brennweite der Sammellinse beträgt $\bar{f} = 4.89\pm0.13\si{\centi\metre}$.\\
Die tatsächliche Brennweite liegt bei $\SI{5}{\centi\metre}$.
Das entspricht einer Abweichung von ca. $2.2 \%$, wenn die Standartabweichung ignoriert wird.

\subsection{Bestimmung der Brennweite}
Der berechnete Wert der Brennweite für die unbestimmte Linse liegt bei $\bar{f} = 9.77\pm0.12\si{\centi\metre}$,
sodass anzunehmen ist, dass diese Linse eine Brennweite von $\SI{10}{\centi\metre}$ haben muss.

\subsection{Methode von Bessel}
Es ergab sich eine ungenaue Brennweite von $f = 11\pm 10\si{\centi\metre}$. 
Diese ist nur korrekt, wenn der Fall $11-10$ betrachtet wird.
Denn die Linse hat eine tatsächliche Brennweite von $\SI{0.5}{\centi\metre}$.

\subsection{Methode von Abbe}
Für die Brennweite des Linsensystems ergab sich ca. $f = \SI{17.35}{\centi\metre}$.