\section{Durchführung}
\label{sec:Durchführung}

\subsection{Überprüfung der Bragg Bedingung}
Für die Überprüfung der Bragg Bedingung braucht es eine Kupfer-Röntgenröhre, einen LiF-Kristall und ein Geiger-Müller-Zählrohr.
Der LiF-Kristall wird auf einen Winkel von $\SI{14}{\degree}$ gestellt, bei dem der Peak der Röntgenstrahlung liegt.
Das Zählrohr wird aber in einem Winkelbereich von $\SI{26}{\degree}$ bis $\SI{30}{\degree}$ mit dem Winkelzuwachs von $\SI{0.1}{\degree}$ aufgebaut.
In einem Integrationszeitraum von $\Delta t = \SI{5}{\second}$ pro Winkel, werden die Impulse aufgenommen.
Aus den gemessenen Werten soll der Winkel mit dem Maximum ermittelt werden und mit dem theoretischen Wert abgeglichen werden.

\subsection{Analyse eines Emissionsspektrums der Cu-Röntgenröhre}
Das Emissionsspektrum der Cu-Röntgenröhre wird in $\SI{0.1}{\degree}$ Schritten mit der Integrationszeit $\Delta t = \SI{10}{\second}$ aufgenommen.
Es wird von $\SI{8}{\degree}$ bis $\SI{25}{\degree}$ gemessen.
Es soll der Bremsberg dargestellt werden, sowie die charakteristischen Linien beschriftet werden.\\
Aus den Daten soll die Halbwertsbreite der $K_{\alpha}$ und $K_{\beta}$ berechnet werden und mit den Energien der beiden Linien sollen die Abschirmkonstanten $\sigma_{1,2,3}$ bestimmt werden.

\subsection{Absorptionsspektren}
Zwischen der Röntgenröhre und dem LiF-Kristall werden Absorber aus 6 verschiedenen Materialien platziert.
Hierbei beträgt die Integrationszeit $\Delta t = \SI{20}{\second}$.
Aus den Daten soll der Winkel der Absorptionskante ermittelt werden, um daraus die Absorptionsenergie der K-Kante zu ermitteln und damit die Abschirmkonstante jedes Materials.