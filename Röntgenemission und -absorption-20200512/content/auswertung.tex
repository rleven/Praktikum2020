\section{Auswertung}
\label{sec:Auswertung}

\subsection{Überprüfung der Bragg Bedingung}
In \autoref{fig:plot} sind die Messdaten gegen den Zählrohrwinkel aufgetragen.
Dabei liegt das Maximum bei ca. $\SI{28.2}{\degree}$.

\begin{figure}
  \centering
  \includegraphics{bragg.pdf}
  \caption{Hier ist der Verlauf der Impulse im Geiger-Müller-Zählrohr gegen die Ausrichtung dessen aufgetragen. Das Maximum liegt bei $\SI{28.2}{\degree}$.}
  \label{fig:plot}
\end{figure}

Mithilfe von \autoref{eq:bragg} und dem bekannten Glanzwinkel bei $\SI{14}{\degree}$, lässt sich der theoretische Winkel des nächsten Maximums berechnen.
Dafür wird die Beugungsordnung $n = 2$ gesetzt.
\newpage
Hieraus ergibt sich für die theoretische Lage des Maximums:

\begin{align*}
  2d\sin\left(\theta_{max}\right) = 2\lambda\\
  \Leftrightarrow \lambda = d\sin\left(\theta_{max}\right)\\
\end{align*}

Für $n = 1$ bei $\SI{14}{\degree}$ ist die Formel:

\begin{align*}
  \lambda= 2d\sin\left(\SI{14}{\degree}\right)
\end{align*}

Daher ergibt sich :

\begin{align*}
  \theta_{max} = \arcsin\left(2\cdot \sin\left(\SI{14}{\degree}\right)\right)\\
  \Leftrightarrow \theta_{max} = \SI{28.94}{\degree}
\end{align*}

Der theoretische Wert für die Lage des Maximums ist somit bei $\theta = \SI{28.94}{\degree}$.

\subsection{Analyse eines Emissionsspektrums der Kupfer-Röntgenröhre}

\begin{figure}[htbp]
  \centering
  \includegraphics{emiss.pdf}
  \caption{Die Messwerte des Emissionsspektrums von Kupfer mit dargestelltem Bremsberg und den charakteristischen Linien.}
  \label{fig:emiss}
\end{figure}
Die minimale Wellenlänge des Bremsberges kann nicht aus dem Graph entnommen werden, da der Bremsberg nicht vollständig dargestellt ist.

Die Halbwertsbreiten sind in \autoref{fig:emiss} dargestellt.\\
Die Halbwertsbreite von $K_{\beta}$ liegt bei ca $\theta_1 = \SI{20.05}{\degree}$, $\theta_2 = \SI{20.55}{\degree}$ und bei $799.5\ \frac{Imp}{s}$.\\
Die Halbwertsbreite von $K_{\alpha}$ liegt bei ca $\theta_3 = \SI{22.35}{\degree}$, $\theta_4 = \SI{22.85}{\degree}$ und bei $2525\ \frac{Imp}{s}$.

\begin{align*}
  \text{Für } K_{\alpha} = \mid \theta_3\ -\ \theta_4 \mid = \SI{0.5}{\degree}\\
  \text{Für } K_{\beta} = \mid \theta_1\ -\ \theta_2 \mid = \SI{0.5}{\degree}
\end{align*}

Die Literaturwerte für die Energien der $K_{\alpha}$ und $K_{\beta}$ Linien betragen\cite{kline}:

\begin{itemize}
  \item $E_{K_{\alpha}} = \SI{8.038}{\kilo\electronvolt}$
  \item $E_{K_{\beta}} = \SI{8.905}{\kilo\electronvolt}$
\end{itemize}

Um das Auflösungvermögen aus \autoref{eq:aufloes} zu berechnen, müssen die Energien der Strahlung bei $\theta_{1,2,3,4}$ bestimmt werden.\\
Aus \autoref{eq:bragg} werden aus den Winkeln die zugehörigen Wellenlängen, die mit $E = h\cdot \nu$ in Energien umgewandelt werden können.
Daraus ergeben sich zwei Halbwertsbreitenergien:

\begin{align*}
  \Delta E_{FWHM,\ \alpha} = \mid E_4\ -\ E_3 \mid = \mid \SI{7.927}{\kilo\electronvolt}\ -\ \SI{8.095}{\kilo\electronvolt} \mid = \SI{0.168}{\kilo\electronvolt}\\
  \Delta E_{FWHM,\ \beta} = \mid E_2\ -\ E_1 \mid = \mid \SI{8.769}{\kilo\electronvolt}\ -\ \SI{8.978}{\kilo\electronvolt} \mid = \SI{0.209}{\kilo\electronvolt}
\end{align*}

Hieraus ergeben sich die Auflösungsvermögen:

\begin{align*}
  A_{\alpha} = \frac{E_{K_{\alpha}}}{\Delta E_{FWHM,\ \alpha}} = \frac{\SI{8.038}{\kilo\electronvolt}}{\SI{0.168}{\kilo\electronvolt}} = 47.845\\
  A_{\beta} = \frac{E_{K_{\beta}}}{\Delta E_{FWHM,\ \beta}} = \frac{\SI{8.905}{\kilo\electronvolt}}{\SI{0.209}{\kilo\electronvolt}} = 42.608
\end{align*}

Um die Abschirmkonstanten $\sigma_{1,2,3}$ zu berechnen, wird die Absorptionsenergie der K-Linie gebraucht.
Sie beträgt bei Kupfer ca. $E_{abs} = \SI{8.98}{\kilo\electronvolt}$\cite{nist}.\\
Mit \autoref{sigma1} werden die Abschirmkonstanten berechnet:

\begin{align*}
  \sigma_1 = 29\ -\ 25.696 = 3.304\\
  \sigma_2 = 29\ -\ 16.644 = 12.356\\
  \sigma_3 = 29\ -\ 7.039 = 21.961
\end{align*}
\newpage
\subsection{Analyse der Absorptionsspektren}
Für die gegebenen Materialien sind folgende Literaturwerte ermittelt worden\cite{nist}:

\begin{table}
  \centering
  \caption{Daten der sechs Materialien\cite{nist}.}
  \begin{tabular}{c c c c c}
    \toprule
    $ $ & $Z$ & $E_K^{Lit}[\si{\kilo\electronvolt}]$ & $\theta_K^{Lit}[\si{\degree}]$ & $\sigma_K$\\
    \midrule
    Zn & 30 & $9.661$ & 18.58 & 3.550\\ \hline
    Ge & 32 & $11.104$ & 16.09 & 3.671\\ \hline
    Br & 35 & $13.471$ & 13.21 & 3.847\\ \hline
    Rb & 37 & $15.203$ & 11.68 & 3.941\\ \hline
    Sr & 38 & $16.107$ & 11.02 & 3.991\\ \hline
    Zr & 40 & $17.996$ & 9.85 & 4.095\\
    \bottomrule
  \end{tabular}
  \label{tab:litdata}
\end{table}

In \autoref{fig:six} sind die Daten aller Materialien aufgetragen, sowie deren Intensitätsminimum und -maximum.

\begin{figure}[htbp]
  \centering
  \includegraphics{six.pdf}
  \caption{Die Messwerte für die sechs Materialien sind hier dargestellt. Die blauen Punkte sind das jeweilige Minimum oder Maximum.}
  \label{fig:six}
\end{figure}

Die Intensitätsmaxima und -minima und das sich daraus ergebende Intensitätsmittel 

\begin{equation}
  I_K = I_K^{min}\ +\ \frac{I_K^{max}\ -\ I_K^{min}}{2}
\end{equation}

sind zusammen mit dessen zugehörigem Winkel und der sich nach \autoref{eq:bragg} daraus ergebenden Absorptionsenergie in \autoref{tab:messdata} aufgetragen.

\begin{table}
  \centering
  \caption{Berechnete Intensitäten, Energien und Winkel der sechs Materialien.}
  \begin{tabular}{c c c c c c c c}
    \toprule
    $ $ & $Z$ & $I_K^{min}\ [Imp/s]$ & $I_K^{max}\ [Imp/s]$ & $I_K\ [Imp/s]$ & $\theta_K[\si{\degree}]$ & $E_{K,abs}\ [\si{\kilo\electronvolt}]$ & $\sigma_K$\\
    \midrule
    Zn & 30 & 54 & 102 & 78 & 18.65 & 9.625 & 3.600\\ \hline
    Ge & 32 & 66 & 122 & 94 & 17.35 & 10.322 & 4.705\\ \hline
    Br & 35 & 9 & 27 & 18 & 13.2 & 13.48 & 3.836\\ \hline
    Rb & 37 & 10 & 64 & 37 & 11.75 & 15.115 & 4.039\\ \hline
    Sr & 38 & 40 & 196 & 118 & 11.09 & 16.002 & 4.105\\ \hline
    Zr & 40 & 112 & 301 & 206.5 & 9.96 & 17.796 & 4.301\\
    \bottomrule
  \end{tabular}
  \label{tab:messdata}
\end{table}