\newpage
\section{Auswertung}
\label{auswertung}

\subsection{Berechnung des Nulleffekts}
Es wurden sieben Messungen des Nulleffekts durchgeführt.

\begin{table}
    \centering
    \caption{Messdaten des Nulleffekts.}
    \begin{tabular}{c|c|c|c|c|c|c|c}
        $N_0$ & 129 & 143 & 144 & 136 & 139 & 126 & 158\\
    \end{tabular}
\end{table}

Aus diesen Werten ergibt sich ein Durchschnitt von $\bar{N_0} = 139 \pm 11$.\\
Die Unsicherheiten werden gemäß der Gauß'schen Fehlerfortpflanzung berechnet.
Der Mittelwert der Messungen lässt sich mit
\begin{equation}
    \overline{v}=\frac{1}{N} \sum_{j=1}^N v_j
\end{equation}
berechnen und die Unsicherheiten dementsprechend mit
\begin{equation}
    \Delta v = \sqrt{\frac{1}{N(N-1)} \sum_{j=1}^N (v-\overline{v})^2}.
\end{equation}

\(v_j\) sind dabei die einzelnen Messungen und \(N\) die Anzahl der Messungen.

Falls Unsicherheiten auftreten, so wird die Gauß'sche Fehlerformel benutzt, um die resultierende Unsicherheit zu berechnen:
\begin{equation}
    \Delta f = \sqrt{\sum_{i=1}^N  \Bigl(\frac{\partial f}{\partial x_i}\Bigr)^2 (\Delta x_i)^2}
    \label{eq:gauß}
\end{equation}

\subsection{Halbwertszeit von Vanadium}
Um doppelte Fehler zu den Messwerten zu vermeiden, wird der Nulleffekt auf $N_0 = 139$ approximiert.\\
Es muss beachtet werden, dass die Impulse in Zeitintervallen von $t = \SI{30}{\second}$ gemessen wurden,
der Nulleffekt jedoch bei $t = \SI{300}{\second}$.
Dies bedeutet, dass bevor der Nulleffekt abgezogen wird, dieser noch durch 10 geteilt werden muss.\\
Die Fehlerverteilung ist Poisson-verteilt, sodass in \autoref{fig:vanadium} die Impulse pro Sekunde gegen die Zeit aufgetragen sind.

\begin{figure}[htbp]
    \centering
    \includegraphics{plot.pdf}
    \caption{Impulse pro Sekunde von Vanadium in Abhängigkeit von der Zeit.}
    \label{fig:vanadium}
\end{figure}
\newpage
In \autoref{fig:vanadium} sind zwei Ausgleichsgeraden eingezeichnet.
Die rote Ausgleichsgerade ist ein Fit über alle Messwerte und deswegen ungenau, da besonders bei niedrigen Messwerten der Nulleffekt für eine große Ungenauigkeit sorgt, wie im Plot zu sehen ist.
Deswegen wurde eine zweite Ausgleichsgerade eingefügt, die lediglich den Bereich bis zur doppelten Halbwertszeit abdeckt.\\
Die Berechnung der Halbwertszeit erfolgt nach \autoref{eq:hwz} und \autoref{eq:gauß}, wobei die Zerfallskonstante $\lambda$ dem Betrag der Steigung der Ausgleichsgerade entspricht.\\
Die Halbwertszeit über alle Messwerte beträgt:
\begin{align*}
    T^{all}_{1/2} = 218\pm 12\si{\second}
\end{align*}
Sobald nur der Bereich bis ca. $\SI{440}{\second}$ betrachtet wird, ändert sich die Halbwertszeit geringfügig:
\begin{align*}
    T^{cut}_{1/2} = 215\pm 16\si{\second}
\end{align*}

\subsection{Halbwertszeit von Rhodium}
Zur Bestimmung der verschiedenen Halbwertszeiten von \ce{^{104}Rh} wird zunächst wie auch bei Vanadium der Nulleffekt von den Messwerten abgezogen.
Es werden wieder die Impulse pro Sekunde betrachtet.\\
In \autoref{fig:rhodium} sind die Messdaten halblogarithmisch aufgetragen.

\begin{figure}[htbp]
    \centering
    \includegraphics{plot1.pdf}
    \caption{Messwerte der Zerfallimpulse von \ce{^{104}Rh}, sowie die Ausgleichsgerade für den langsamen Zerfall und den kurzen Zerfall, ebenso die errechnete Kurve aus beiden Ausgleichsgeraden.}
    \label{fig:rhodium}
\end{figure}

In diesem Plot ist die blaue Ausgleichsgerade der langsame Zerfall und die grüne Ausgleichsgerade der kurze Zerfall.
Daraus lassen sich die verschiedenen Halbwertszeiten berechnen, die Rhodium hat:
\begin{align*}
    T^{short}_{1/2} = 36.7\pm 1.4\si{\second}\\
    T^{long}_{1/2} = 217\pm 29\si{\second}
\end{align*}
Aus beiden Geraden errechnet sich eine Kurve die mit den Messwerten übereinstimmt.