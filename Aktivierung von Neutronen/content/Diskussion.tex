\newpage
\section{Diskussion}
\label{Diskussion}

\subsection{Halbwertszeit von Vanadium}
Der Theoriewert für die Halbwertszeit von \ce{^{52}V} liegt bei $T^{theo}_{1/2} = \SI{224.6}{\second}$\cite{vana}.
Der aus den Messwerten berechneten Wert liegt bei $T^{cut}_{1/2} = 215\pm 16\si{\second}$.\\
Das entspricht einer Abweichung von ca. $4.27\%$.
\subsection{Halbwertszeiten von Rhodium}
Die Theoriewerte für den langsamen und den schnellen Zerfall von \ce{^{104}Rh} liegen bei $T^{short}_{1/2} = \SI{42.3}{\second}$\cite{rh1} und $T^{long}_{1/2} = \SI{260}{\second}$\cite{rh2}.
Aus den Messwerten ergeben sich die Halbwertszeiten:
\begin{align*}
    T^{short}_{1/2} = 36.7\pm 1.4\si{\second}\\
    T^{long}_{1/2} = 217\pm 29\si{\second}
\end{align*}
Im Falle des schnellen Zerfalls beträgt die Abweichung $13.24\pm 3.31\%$.\\
Im Falle des langsamen Zerfalls beträgt die Abweichung $16.54\pm 11.15\%$.