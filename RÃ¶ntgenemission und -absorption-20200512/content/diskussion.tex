\newpage
\section{Diskussion}
\label{sec:Diskussion}

\subsection{Bragg-Bedingung}
Die theoretische Lage des Maximums in \autoref{fig:plot} liegt bei $\theta_{theo} = \SI{28.94}{\degree}$.
Der gemessene liegt bei $\theta = \SI{28.2}{\degree}$.\\
Das entspricht einer relativen Abweichung von $\SI{2.56}{\percent}$.

\subsection{Absorptionsspektren}
In \autoref{tab:discuss} sind die Literaturwerte der Absorptionsenergie, die Berechneten und die relative Abweichung dargestellt.

\begin{table}
    \centering
    \caption{Tabelle mit den theoretischen Absorptionsenergien, sowie den Berechneten un dessen relative Abweichung.}
    \begin{tabular}{c c c c}
        \toprule
        $ $ & $E_{abs}^{Lit}\ [\si{\kilo\electronvolt}]$ & $E_{abs}\ [\si{\kilo\electronvolt}]$ & Abweichung [$\si{\percent}$]\\
        \midrule
        Zn & 9.661 & 9.625 & 0.37\\ \hline
        Ge & 11.104 & 10.322 & 7.04\\ \hline
        Br & 13.471 & 13.48 & 0.07\\ \hline
        Rb & 15.203 & 15.115 & 0.58\\ \hline
        Sr & 16.107 & 16.002 & 0.65\\ \hline
        Zr & 17.996 & 17.796 & 1.11\\
    \bottomrule
    \end{tabular}
    \label{tab:discuss}
\end{table}