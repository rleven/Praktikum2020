\section{Auswertung}
\label{sec:Auswertung}

Für drei Erd-Verbindungen wurden jeweils drei Messungen vorgenommen.\\
In \autoref{tab:tab1} sind die gemessenen Widerstände und Brückenspannungen aufgelistet.
Die verwendete Speisespannung betrug $\SI{0.72}{\volt}$ bei allen Messungen.
Die Spannungen sind jeweils in \si{\milli\volt} und die Widerstände in \si{\milli\ohm}.

\begin{table}
  \centering
  \caption{Messdaten der drei seltenen Erden.}
  \resizebox{\textwidth}{!}{%
  \begin{tabular}{c|c|c|c|c|c|c|c|c|c|c|c}
    $U_{Br}^{leer}$ & $R^{leer}$ & $U_{Br}^{Dy}$ & $R^{Dy}$ & $U_{Br}^{leer}$ & $R^{leer}$ & $U_{Br}^{Gd}$ & $R^{Gd}$ & $U_{Br}^{leer}$ & $R^{leer}$ & $U_{Br}^{Nd}$ & $R^{Nd}$\\
    \midrule
    2.25 & 3097.5 & 3.5 & 1555 & 2.4 & 3077.5 & 2.3 & 2315 & 2.65 & 3085 & 2.6 & 2978.5\\
    2.3 & 3112.5 & 3.5 & 1555 & 2.5 & 3100 & 2.3 & 2315 & 2.65 & 3085 & 2.65 & 2995\\
    2.35 & 3087.5 & 3.5 & 1537.5 & 2.6 & 3082.5 & 2.25 & 2315 & 2.7 & 3085 & 2.6 & 2990\\
  \end{tabular}}
  \label{tab:tab1}
\end{table}

Nach \autoref{eq:chir} ergeben sich daraus folgende Werte für $\chi_R$:
\begin{table}
  \centering
  \caption{Experimentelle Werte für $\chi_R$ der drei Erden, über die Widerstände berechnet.}
  \label{tab:tab2}
  \begin{tabular}{c|c|c}
    $\chi_{Dy}$ & $\chi_{Gd}$ & $\chi_{Nd}$\\
    \midrule
    0.0240 & 0.0176 & 0.0012\\
    0.0243 & 0.0181 & 0.0010\\
    0.0241 & 0.0177 & 0.0010\\
  \end{tabular}
\end{table}

Die Formel zur Berechnung des Mittelwerts lautet:

\begin{equation}
    \bar{x} = \frac{1}{N} \sum_{i=1}^N x_i
\end{equation}

Der Fehler des Mittelwerts lautet:

\begin{equation}
    \Delta x = \frac{1}{\sqrt{N}}\sqrt{\frac{1}{N-1}\sum_{i=1}^N (x_i-\bar{x})^2}
    \label{eq:std}
\end{equation}

Daraus ergeben sich die Mittelwerte mit Fehler, laut \autoref{eq:std}:
\begin{align*}
  \bar{\chi}_{Dy} = 0.02413\pm 0.00015\\
  \bar{\chi}_{Gd} = 0.01780\pm 0.00026\\
  \bar{\chi}_{Nd} = 0.00107\pm 0.00012
\end{align*}

Eine weitere Möglichkeit ist $\chi$ über die Speisespannung und die Brückenspannung zu ermitteln, nach \autoref{eq:chiu}.
In \autoref{tab:tab3} sind die Werte für $\chi_U$ eingetragen.

\begin{table}
  \centering
  \caption{Experimentelle Werte für $\chi_U$ der drei Erden, über die Brückenspannung berechnet.}
  \label{tab:tab3}
  \begin{tabular}{c|c|c}
    $\chi_{Dy}$ & $\chi_{Gd}$ & $\chi_{Nd}$\\
    \midrule
    0.1511 & 0.1474 & 0.0793\\
    0.1468 & 0.1474 & 0.0808\\
    0.1425 & 0.1442 & 0.0793\\
  \end{tabular}
\end{table}

Dessen Mittelwerte sind:
\begin{align*}
  \bar{\chi}_{Dy} = 0.147\pm 0.004\\
  \bar{\chi}_{Gd} = 0.1463\pm 0.0018\\
  \bar{\chi}_{Nd} = 0.0798\pm 0.0009
\end{align*}

Die theoretischen Werte berechnen sich über \autoref{eq:chitheo} zu\footnote[1]{Die Werte zur Masse und Anzahl der Momente N wurden hier\cite{v606} entnommen.}:
\begin{align*}
  \chi_{Dy}^{theo} = 0.0257\\
  \chi_{Gd}^{theo} = 0.0146\\
  \chi_{Nd}^{theo} = 0.0027
\end{align*}