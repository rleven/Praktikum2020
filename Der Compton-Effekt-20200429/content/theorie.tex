\section{Theorie}
\label{sec:Theorie}

\subsection{Der Compton-Effekt}

Der Compton-Effekt beschreibt das physikalische Prinzip, wenn ein ausreichend energetisches Photon der Wellenlänge $\lambda_1$ auf ein Atom prallt und bei dem inelastischen Stoß mit einer niedrigeren Energie weiterfliegt.
Dazu muss das Photon auf ein schwach gebundenes Elektron treffen, dem es einen Teil seiner Energie abgibt.
Die Austrittswellenlänge wäre dann $\lambda_2$, welche zusätzlich um den Winkel $\theta$ abgelenkt wird.\\
Die Gleichung für die Energie des Photons (in $eV$) lautet:

\begin{equation}
    E_{Photon} = \frac{h\cdot c}{\lambda \cdot 1.6022\cdot 10^{-19}}
    \label{eq:enrgy}
\end{equation}

Der Unterschied der Wellenlänge jenes Photons $\Delta \lambda = \lambda_2\ -\ \lambda_2$ kann auch über den Austrittswinkel $\theta$ bestimmt werden:

\begin{equation}
    \Delta \lambda = \frac{h}{m_e\cdot c}(1\ -\ \cos{\theta})
    \label{eq:compton}
\end{equation}

Hierbei ist der Vorfaktor $\frac{h}{m_e\cdot c}$ die sogenannte \textit{Compton-Wellenlänge} $\lambda_C$.\\
Die Wellenlängendifferenz $\Delta \lambda$ wird nach \autoref{eq:compton} demnach bei $\theta = \pi$, also $180°$ maximal, nämlich $2\cdot \lambda_C$.

\subsection{Erzeugung der Röntgenstrahlung}

Die charakteristische Röntgenstrahlung entsteht durch den Aufprall energiereicher Elektronen auf ein bestimmtes Material.
Dazu werden die Elektronen von einer Glühkathode auf eine Anode, in einer evakuierten Umgebung, aus dem nötigen Material beschleunigt.
Da energiereiche Elektronen ionisierende Strahlung sind, ionisieren sie das Anodenmaterial.
Das führt dazu, dass ein energiereiches Elektron im Atom in einen energieärmeren Zustand versetzt wird, also in eine innere Schale wandert.
Dieser Energieverlust wird als charakteristische Röntgenstrahlung emittiert. 
Deshalb besteht ein Spektrum der Strahlung aus klaren, scharfen Linien.
\\
\\
Im selben Spektrum findet sich aber auch ein kontinuierliches Spektrum von Röntgenlicht.
Dies entsteht wegen der Bremsstrahlung, welche das beschleunigte Elektron auslöst, wenn es ins Coulomb-Feld des Atoms gerät und abgebremst wird.
Da das Elektron bei diesem Prozess einen Teil, bis hin zur Gesamtenergie, in Form von Strahlung abgeben kann, wird ein kontinuierliches Spektrum emittiert.

\subsection{Die Bragg'sche Bedingung}

Um das Röntgenlicht zu analysieren, wird die Bragg'sche Reflexion verwendet.\\
Hierbei werden die Photonen unter einem bestimmten Winkel $\alpha$ an einen Gitterkristall gesendet, sodass konstruktive Interferenz entsteht.
Zusammen mit der Gitterkonstante $d$ und der Beugungsordnung $n$, lässt sich damit die Wellenlänge nachweisen.
Die Formel der Bragg'schen Bedingung lautet:

\begin{equation}
    2d\sin{\alpha} = n\cdot \lambda
    \label{eq:bragg}
\end{equation}

\subsection{Totzeit im Geiger-Müller-Zählrohr}

Das Geiger-Müller-Zählrohr besteht aus einem Zylinder gefüllt mit Gas, einer Wandkathode und einer Stabanode.
Wenn ionisierende Strahlung mit dem Gas in Berührung kommt, werden Elektronen freigesetzt, die die Anode aufnimmt und als Signal weitergibt.
Sind allerdings alle Gasatome angeregt, so wird die Intensität der Strahlung verfälscht. 
Die Zeit bis sich die Gasatome wieder neutralisieren, nennt man \textit{Totzeit} $\tau$.
Zur Behebung des Fehlers reicht die Gleichung:

\begin{equation}
    I = \frac{N}{1\ -\ \tau N}
    \label{eq:deadtime}
\end{equation}

\subsection{Gauß'sche Fehlerfortpflanzung}

Die Formel für die Gauß'sche Fehlerfortpflanzung einer Messgröße $x$ mit Fehler $\Delta x$ lautet:

\begin{equation}
    y(x\ +\ \Delta x) = y(x)\ +\ \frac{dy}{dx}\cdot \Delta x = y\ +\ \Delta y
    \label{eq:gauss}
\end{equation}